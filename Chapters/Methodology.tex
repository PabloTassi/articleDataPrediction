\documentclass[main-singleColumn.tex]{subfiles}
\begin{document}
%----------------------------------------------------------------------------------
%---------------------------------- INTRO -----------------------------------------
%----------------------------------------------------------------------------------
\section{POD-PCE based predictor}
\label{section:methodology}
\subsection{The coupling}
\label{subsection:predictor}
Let $u(\mathbf{X},t_1)$ be a measured field defined for a time $t_1$ for geographical points $\mathbf{X}=\{x_1,...x_M\}$. Let $(\theta_1,...,\theta_V)$ be a set of measured forcing parameters. We want to elaborate a dynamical model $\mathcal{H}$, that gives an estimation of a future state $u(\mathbf{X},t_2)$ from the knowledge of the past state and the estimation of the forcing parameters on the  time interval $[t_1,t_2]$, as in Eq. \ref{eq:prediction:model}.

\begin{equation}
  u(\mathbf{X},t_2) = \mathcal{H} (u(\mathbf{X},t_1),t_2-t_1,\theta_1,...,\theta_V)
  \label{eq:prediction:model}
\end{equation}

We present hereby a methodology for constucting a data-based model $\mathcal{H}$, using a coupling between POD and PCE. In fact, as we are able to project the future state on the POD basis as follows:

\begin{equation}
  u(\mathbf{X},t_2) \approx \sum_{k=1}^{d} v_k(t_2) \phi_k(\mathbf{X}) \qquad ,
\end{equation}

Or, in the KLT formulation, as follows:
\begin{equation}
  u(\mathbf{X},t_2) \approx \sum_{k=1}^{d} a_k(t_2) \sqrt{\frac{\lambda_k}{N}} \phi_k(\mathbf{X}) \qquad ,
\end{equation}

Then for each $1 \leq k \leq d$, we could construct a dynamical model $H_k$ for $a_k$ as in  Eq. \ref{eq:prediction:minimodel}, using PCE.

\begin{equation}
  a_k(t_2) = H_k (a_k(t_1),t_2-t_1,\theta_1,...,\theta_V)
  \label{eq:prediction:minimodel}
\end{equation}

Therefore, the model $\mathcal{H}$ in  Eq. \ref{eq:prediction:model} would be written as in Eq. \ref{eq:prediction:fullmodel}.

\begin{equation}
 \mathcal{H} (u(\mathbf{X},t_1),t_2-t_1,\theta_1,...,\theta_V) \approx \sum_{k=1}^{d} H_k (a_k(t_1),t_2-t_1,\theta_1,...,\theta_V) \sqrt{\frac{\lambda_k}{N}} \phi_k(\mathbf{X})
  \label{eq:prediction:fullmodel}
  \end{equation}


\subsection{Adapted sensitivity analysis}

The methodology presented in Subsection \ref{subsection:sensitivity} is adapted for the analysis of each model $H_k$, but does not put the contributions of each model's inputs in perspective with those of the other modes. \\

In order to rank all the polynomial contributions that are involved in the prediction of the final field $u(\mathbf{X},t)$, via each PCE model $H_k$, the calculation of generalized weights is proposed. \\

For the construction of the each model $H_k$ that approximates the temporal signal $a_k$, let $\left\{ \zeta_{\underline{\alpha}}^k \right\}_{|\underline{\alpha}| \leq P_k}$ be the set of multivariate polynomials deduced on the input variables $(a_k(t_1),t_2-t_1,\theta_1,...,\theta_V)$. These multivariate polynomials are multiplied by the coefficients $\left\{ c_{\underline{\alpha}}^k \right\}_{|\underline{\alpha}| \leq P_k}$ in the model deduced using LARS algorithm, with $P_k$ the related polynomial degree. Each model $H_k$ can be written as $\sum_{|\underline{\alpha}| \leq P_k} c_{\underline{\alpha}}^k  \zeta_{\underline{\alpha}}^k $. Therefore, the final model $\mathcal{H}$ as in Eq. \ref{eq:prediction:fullmodel} reads:
\begin{equation}
   \mathcal{H} (u(\mathbf{X},t_1),t_2-t_1,\theta_1,...,\theta_V) \approx \sum_{k=1}^{d} a_k(t_1) \sqrt{\frac{\lambda_k}{N}} \phi_k(\mathbf{X}) \approx \sum_{k=1}^{d} \left( \sum_{|\underline{\alpha}| \leq P_k} c_{\underline{\alpha}}^k  \zeta_{\underline{\alpha}}^k \right)  \sqrt{\frac{\lambda_k}{N}} \phi_k(\mathbf{X})
  \end{equation}

In Eq. \ref{eq:sensitivity:weights}, the contribution of each $ \zeta_{\underline{\alpha}}^k$ in $H_k$, and therefore in the prediction of $a_k$, is calculated via the weights calculated as follows:
\begin{equation}
  w_{\zeta_{\underline{\alpha}}^k} = \dfrac{|c_{\underline{\alpha}}^k|}{ \sum_{|\underline{\beta}| \leq P} |c_{\underline{\beta}}^k|}
\end{equation}

As the families $\{a_k\}_{k=1}^d$ and $\{\phi_k\}_{k=1}^d$ are bi-orthonormal, implying that $\forall k \in \{1,...,d\} \quad ||a_k||_{\mathbb{R}^N}=1$ and $||\phi_k||_{\mathbb{R}^M}=1$, the weights $w_{\zeta_{\underline{\alpha}}^k}$ can be calibrated by the multiplicative coefficients $\sqrt{\lambda_k/N}$. We therefore propose the computation of generalized weights as:
\begin{equation}
  \label{eq:sensitivity:generalizedWeights}
  W_{\zeta_{\underline{\alpha}}^k} = \dfrac{w_{\zeta_{\underline{\alpha}}^k}*\sqrt{\frac{\lambda_k}{N}}}{ \sum_{e=1}^d \sqrt{\frac{\lambda_e}{N}}} = \dfrac{|c_{\underline{\alpha}}^k|\sqrt{\frac{\lambda_k}{N}}}{ \left( \sum_{|\underline{\beta}| \leq P} |c_{\underline{\beta}}^k| \right) \left(\sum_{e=1}^d \sqrt{\frac{\lambda_e}{N}}\right)}
\end{equation}

And we have:
\begin{equation}
  \begin{tabular}{m{2.5cm} m{10cm}}
    $\sum_{k=1}^d \sum_{|\underline{\alpha}| \leq P} W_{\zeta_{\underline{\alpha}}^k}$ & $= \sum_{k=1}^d \sum_{|\underline{\alpha}| \leq P} \left(  \dfrac{|c_{\underline{\alpha}}^k|\sqrt{\frac{\lambda_k}{N}}}{ \left( \sum_{|\underline{\beta}| \leq P} |c_{\underline{\beta}}^k| \right) \left(\sum_{e=1}^d \sqrt{\frac{\lambda_e}{N}}\right)}\right)$ \\
    & \\
    & $= \sum_{k=1}^d \left(\dfrac{\sqrt{\frac{\lambda_k}{N}}}{\left( \sum_{|\underline{\beta}| \leq P} |c_{\underline{\beta}}^k| \right) \left(\sum_{e=1}^d \sqrt{\frac{\lambda_e}{N}}\right)} \left( \sum_{|\underline{\alpha}| \leq P}|c_{\underline{\alpha}}^k| \right) \right)$ \\

    %$=  \sum_{k=1}^d \dfrac{\sqrt{\frac{\lambda_k}{N}}{ \left( \sum_{|\underline{\beta}| \leq P} |c_{\underline{\beta}}^k| \right) \left(\sum_{e=1}^d \sqrt{\frac{\lambda_e}{N}}\right)}  \sum_{|\underline{\alpha}| \leq P}|c_{\underline{\alpha}}^k| $\\
    & $= 1$
  \end{tabular}     
\end{equation}

Which allows to rank the multivariate polynomials alltogether.


\subsection{Accuracy estimation}
\label{subsection:accuracy}
There are two determining parts in the POD-PCE learning process. First, the PCE calibration $H_k(t)$ of each mode $a_k(t)$ should be as accurate as possible. Second the reconstructed field $\sum_{k=1}^{d} H_k(t) \sqrt{\frac{\lambda_k}{N}} \phi_k(\mathbf{X})$ for a given rank $d$ should be as close to the real field $Z(x,t)$ as possible.

In order to check the distance between each mode and its PCE approximate, a \textit{generalization error} can be calculated as:
\begin{equation}
  \delta = \mathbb{E} \left[ (a_k(t) - H_k(t))^2 \right]
  \end{equation}

This error, as explained by \cite{Blatman2009} can be estimated as in Eq. \ref{eq:empiricalError}, called the \textit{empirical error}, where $N$ is the number of realizations of the output variable $a_k$, here the POD mode of rank $k$. 
\begin{equation}
  \label{eq:empiricalError}
  \delta \approx \delta_{emp} = \dfrac{1}{N} \sum_{j=1}^N \left( a_k(t_j) - H_k(t_j) \right)^2 
  \end{equation}

Its relative estimate can be calculated as in Eq. \ref{eq:relativeEmpiricalError}.
\begin{equation}
  \label{eq:relativeEmpiricalError}
  \epsilon \approx \dfrac{\delta_{emp}}{\mathbb{V}[a_k]}
\end{equation}

For the final prediction phase, the distance between reality $Z$ and the POD-PCE approximation $\mathcal{H}$ at time $t_j$ can be estimated by the relative Root Mean Squared Error (RMSE) as in Eq \ref{eq:relativeRMSE} where $(x_i)_{i=1,...,M}$ are the space coordinates over which the real field is defined.
\begin{equation}
  \label{eq:relativeRMSE}
    RMSE(Z,\mathcal{H})(t_j) = \sum_{i=1}^M \dfrac{(Z(x_i,t_j)-\mathcal{H}(x_i,t_j))^2}{(Z(x_i,t_j))^2}
\end{equation}

The mean error in time can therefore be estimated as:
\begin{equation}
  \label{eq:timeAveragedRelativeRMSE}
    \text{time Averaged - } RMSE(Z,\mathcal{H}) = \dfrac{1}{N} \sum_{j=1}^N RMSE(Z,\mathcal{H})(t_j)
\end{equation}



%\subsection{Methodology}

%\subsection{Algorithm illustration}
%separate set to : training set + prediction set \\

%Learning step on training set\\
%1 - perform POD and store POD basis \\
%2 - learn one PCE per temporal coefficient\\

%Qualification step on prediction set\\
%2 - For each measurement : estimate prediction of temporal coefficient using associated PCE \\
%3 - Reconstitute field prediction: multiply coefficient by POD basis element and sum\\
%4 - Accuracy analysis: Compare to real field\\

%Comment on training set size selection. 

%\subsection{Discussion about accuracy analysis}
%\subsubsection{Errors calculation}
%---------------------------------------------------------------------------

\end{document}
