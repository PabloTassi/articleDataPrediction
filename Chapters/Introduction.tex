\documentclass[main-singleColumn.tex]{subfiles}
\begin{document}
%----------------------------------------------------------------------------------
%---------------------------------- INTRO -----------------------------------------
%----------------------------------------------------------------------------------
\section{Introduction}

Machine Learning technics (ML) and their application to concrete problems have made a promising takeoff in the last years. This has been particularly the case for fields where the measurement potential has increased, and the numerical modeling approaches still not unanime. Our example of interest takes place in the geosciences community, around the modeling of the solid particles dynamics (sediments dynamics) due to the action of a fluid flow, particularly free surface flows in natural environments (hydrodynamics). The settling of the sediments in the bottom is what forms the underwater topography (bathymetry) and makes its patterns evolve (morphodynamics). The bottom elevation information is capital for many applications, precisely for the prediction of the resulting flow. More specifically, this work is conducted in an industrial context where historical monitoring data of sediment dynamics are available for a coastal channel that is subject to silting. The issue is concrete, involving a complex physical problem with high non linearities and interactions (waves and tides forcing). \\

A recent article \cite{Goldstein2019} give a complete overview of the use of ML in coastal sediment transport modeling. Various methods are used, for eg. Artificial Neural Networks (ANN) and Bayesian Networks (BN), and the interest variables to approximate can differ from the characterization of coastal morphological forms to the estimation of suspended sediment concentration. The great majority of studied problems gravitate around the evolutions of shorelines, which are medium to long term phenomena (years, decades). An application gap is noticed in the study of artificial channels with a forcing that implies fast dynamics (days, weeks), in a medium spatial scale. \\

Among the limitations of ML technics are: no explicit model can be provided in the end, and the calibration of hyperparameters is often necessary. \cite{Torre2019} propose the use of Polynomial Chaos Expansion (PCE) in order to achieve a Machine Learning that overcomes those limitations, where only a polynomial degree has to be selected. In this work, PCE has shown to be as performent as classical ML technics, with the advantage of being simple to write and therefore simple to interpret, as well as offering a probabilistic framework for uncertainties quantification of the output and sensitivity to the various inputs study. Theoretical evidence of the convergence of such expansion towards reality exists,  provided that the optimal basis is selected. Recent developments (\cite{Blatman2009}) defined a framework of sparse PCE which allows to deal with small data, using Least Angle Regression (LAR, or LARS as introducted by \cite{Efron2004}). Using the latter, \cite{Torre2019} show that PCE delivers accurate pointwise predictions on problems defined from analytical functions (for eg. Sobol function), as well as some tested one-dimensional (1D) concrete data. In these examples, the analytical functions allowed a generation of sufficiently large samples (the authors used up to $10^4$ individuals to assess the convergence for example), and the concrete data were minimum of size $500$ (an example where the number of inputs is $12$). The application of interest in the present work allows the use of maximum $60$ measurements with a number of forcing variables that could go up to $15$. Obviously, this \og small data \fg{} configuration is a considerable handicap for the dimension of the problem, especially given that the interest variable is a two-dimension (2D) bathymetry field.  To the authors knowledge, no assessment of PCE use on 2D field data has been proposed in the literature, least of all in the sediment dynamics community, and even less so for coastal morphodynamics. \\

In this work, in order to deal with the 2D aspect of the interest variable, a Proper Orthogonal Decomposition (POD) is perfomed, for pattern recognition and dimensionality reduction. Such methods have been actively used in the last years in the sediment dynamics community. In coastal morphodynamics, joint publications \cite{Larson2003} and \cite{Southgate2003} review the use of linear and non linear decomposition techniques. The works mainly focused, here again, on shoreline evolution study ( e.g. \cite{Medina1992}, \cite{Rozynski2001}), with fewer application to estuaries (e.g. \cite{Karunarathna2008}) and river processes (e.g. \cite{Yilmaz2018}). Some attemps to hybridation with modeling are given by \cite{Karunarathna2008} for example using a diffusion equation, or simpler correlation analysis between the temporal components and forcing variables as in \cite{Miller2007}. No attempt to PCE is given. \\

In the present work, we propose a POD-PCE coupling as a forecasting algorithm, stricly based on field measurements. Some difficulties that are intrisic to field data (incomplete fields, uncontrolled frequencies of measurements, paucity  of the data) are encountered and discussed. Convergence analysis and robustness tests to data choice and input probabilistic characterization (choice of marginals) are proposed. The performence of POD-PCE coupling in the prediction of a 2D field data involving non linear dynamics and based on a particularly small set is assessed, and an adjusted sensitivity analysis is proposed based on the polynomial chaos coefficients. \\ 

After a short introduction to the industrial case and involved physics and data in Section \ref{section:application}, a brief reminder of theoretical elements around POD and PCE is given in Section \ref{section:theory}. In the latter, a simple weight calculation for the assesment of the influence of input variables on an output polynomial model is proposed for sensitivity analysis, and generalized in the case of POD-PCE coupling in Section \ref{section:methodology} where the coupling strategy is introduced. The step-by-step results are shown in Section \ref{section:results} divided as follows: the POD performence and interest are demonstrated in Subsection \ref{subsection:POD}, the PCE learning of the POD's temporal coefficient is assessed in Subsection \ref{subsection:PCE} using different model choices, presented with associated residual, convergence and sensitivity studies. Last, the performence of the POD-PCE predictor is discussed in Subsection \ref{subsection:couplingResult}.
%---------------------------------------------------------------------------

\end{document}
